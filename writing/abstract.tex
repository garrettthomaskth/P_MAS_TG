\section*{\centering Abstract}
Model checking has proven to be a valuable design validation tool to ensure the correctness of hardware and software design. Recently, symbolic model checking using temporal logics has shown great promise in the field of control and task planning synthesis, as it allows for the formulation of complex tasks and provides an automatic and exhaustive search of all possible paths. The major drawback of this method is what is known as the state-space explosion problem. Even with relatively small environments and formulas, the resulting product automaton can become too large to search. Various methods exist which attempt to mitigate the problem, however these methods have limited applicability in the control and task planning synthesis context. We propose a greedy algorithm to address this problem. Our algorithm makes use of a novel distance measure in the B\"uchi automaton corresponding to the specified formula. At each step the optimal path is found which decrease this distance, which together produces an approximation of the globally optimal path. The performance of this algorithm is then analysed on various control planning synthesis examples from the literature and compared to the current accepted algorithm.     


%This problem has been studied extensively from the design validation point of view, and there are various methods to mitigate the problem. These methods however have limited applicability in the control and task planning synthesis context. 



\newpage

\section*{\centering Sammanfattning}

