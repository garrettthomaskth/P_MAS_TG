\section{Search Algorithms}
The task is now to compute a path that satisfies our LTL formula. The current accepted algorithm does an exhaustive search of the product automaton to find the optimal path (again this may not actually be the optimal path \cite{schuppan05}. This however is a computationally intensive task. We present an approximation algorithm that gives a \textit{good} path, but not necessarily the optimal path. This can be attractive if the cost of the path is not of dire importance. We first present the current standard algorithm and then our algorithm. 
\subsection{Accepted Algorithm}
The search algorithm used in many recent works on the specific type of control planning synthesis comes from this prefix-suffix structure. The basic idea is to find a path from an initial node, $q_0$ to an accepting node, $q_f$, and then find a path from the $q_f$ back to itself. The first part from $q_0$ to $q_f$ is the prefix and the second part $q_f$ back to $q_f$ is the suffix. Then the resulting path, $\tau$, will be the prefix, followed by the suffix repeated infinitely many times. This path is thus accepting because the suffix finds the path from an initial state back to itself, and thus contains the initial state, and is repeated infinitely many times $q_f \in \Inf \tau  \Rightarrow \Inf \tau \cap \F \neq \emptyset$. This algorithm, or simple variations of it, are used in many works on motion planning synthesis \cite{fainekos09}, add more, so we will refer to it as the \textit{accepted} algorithm. 

%Algorithm \ref{optrun}, from \cite{guo15}, gives pseudocode of how to compute $R_{opt}$.
%\begin{algorithm}
%\caption{OptRun()}\label{optrun}
%\begin{algorithmic}[1]
%\Require Input $\A_p, S' = \Q_0'$ by default
%\Ensure $R_{opt}$
%%\Procedure{MyProcedure}{}
%%\State If $Q_0'$ or $\F'$ is empty, construct $Q_0'$ or $\F'$ first.
%\State For each accepting state $q_f' \in \F'$, call $\texttt{DijksCycle}(\A_p,q_f')$. 
%\State For initial state $q_0' \in S'$, call $\texttt{DijksTargets}(\A_p,q_0',\F')$.
%\State Find the pair of $(q_{0,opt}',q_{f,opt}')$ that minimizes the total cost
%\State Optimal accepting run $R_{opt}$, prefix: shortest path from $q_{0*}'$ to  $q_{f*}$; suffix: the shortest cycle from $q_{f*}'$ and back to itself.
%\end{algorithmic}
%\end{algorithm}

Algorithm \ref{optrun}, from \cite{guo15}, gives pseudocode of how to compute $R_{opt}$.
\begin{algorithm}
\caption{OptRun()}\label{optrun}
\begin{algorithmic}[1]
\Require Input $\A_p, S' = \Q_0'$ by default
\Ensure $R_{opt}$
%\Procedure{MyProcedure}{}
%\State If $Q_0'$ or $\F'$ is empty, construct $Q_0'$ or $\F'$ first.
\State For each accepting state $q_f' \in \F'$, calculate the optimal path back to $q_f'$. 
\State For initial state $q_0' \in S'$, find the optimal path to each $q_f' \in \F$.
\State Find the pair of $(q_{0,opt}',q_{f,opt}')$ that minimizes the total cost
\State Optimal accepting run $R_{opt}$, prefix: shortest path from $q_{0*}'$ to  $q_{f*}$; suffix: the shortest cycle from $q_{f*}'$ and back to itself.
\end{algorithmic}
\end{algorithm}

Meng Guo has created a public github repository, P-MAS-TG (Planner for Multiple Agent System with Temporal Goals) \cite{pMasGit}. The function \texttt{dijkstra\_plan\_networkX} in \texttt{P\_MAS\_TG\\discrete\_plan.py} is approximately equvialent to Algorithm \ref{optrun}. The work of finding the optimal path from $q_f'$ back to $q_f'$ and from $q_0'$ to all $q_f'$ is done by \texttt{dijkstra\_predecessor\_and\_distance} from the NetworkX python package \cite{schult08}. $\texttt{dijkstra\_predecessor\_and\_distance}(\A_p,q_0)$ returns two dictionaries; one containing a list of all the nodes $q_0$ is a predecessor of and one containing the distances to each of these nodes. When we provide computational examples for the accepted algorithm, we will be using this repository. %The script run to create the examples is included in the appendix.  

%$\texttt{DijksTargets}(\A_p,q_0',\F)$ computes the shortest paths in $\A_p$ from initial state $q_0' \in \Q_0$ to every accepting node in $\F$ using Dijkstra's algorithm \cite{dijkstra59} and $\texttt{DijksCycle}(\A_p,q_f')$ computes the shortest path in $\A_p$ from accepting state $q_f'$ back to itself using Dijkstra's algorithm.

The worst case computational complexity of this algorithm $\mathcal{O}(|\delta'| \cdot \log|\Q'| \cdot (|\Q_0'| + |\F'|))$ because the worst case complexity for a Dijkstra search is $\mathcal{O}(|\delta'| \cdot \log|\Q'| )$ and Algorithm \ref{optrun} does $(|\Q_0'| + |\F'|)$ Dijkstra searches (one for each initial node and one for each accepting node).

\subsection{Our Algorithm}
As we can see, the current algorithm has to do a lot of work. First it has to do Dijkstra's search for each initial state, and then one for each accepting state (the number of accepting states is at least the size of the FTS). The state space that is being searched can also become very big, which is known as the state explosion problem \cite{clarke99}. The size of the product automaton, $|\A_p|$ is the size of the B\"{u}chi automaton corresponding to the LTL formula times the size of the FTS i.e.\ $|\Pi| \dot |\Q|$. The size of the B\"{u}chi automaton corresponding to the LTL formula is then usually exponential in the size of the formula. We can imagine how much searching is needed if we have and FTS and B\"{u}chi that are both fairly large. To solve this problem, we suggest an algorithm that sacrifices optimality but preforms much faster than the current accepted algorithm. 

The idea stems from the fact that $q' = \langle \pi, q \rangle \in \Q'$ is an accepting state of $\A_p$ iff $q \in Q$ is an accepting state of $\A_\varphi$. Thus finding an accepting state in the product automaton is essentially finding an accepting state of the LTL B\"{u}chi automaton. We therefore suggest assigning a distance measure in the LTL B\"{u}chi automaton that carries over to the product automaton. To do this, we first define a B\"{u}chi automaton that includes information on the distance to an accepting state.

\begin{definition}
\label{defBWD}
An NBA with distance, NBAD, is defined by a six-tuple:
\begin{align*}
\mathcal{A}_{\varphi,d} = (\mathcal{Q},2^{AP},\delta,\mathcal{Q}_0,\mathcal{F},d)
\end{align*}
where $\mathcal{Q},2^AP,\delta,\mathcal{Q}_0,\mathcal{F}$ are defined as in definition \ref{defNBA} and $d:\Q \rightarrow \mathbb{Z}$ is defined as 
\begin{align*}
d(q_n) = \min_x \{x | q_x \in \F \text{ and } q_k \in \delta(q_{k-1},S_{k-1}) \text{ for some } S_k \in 2^{AP} \text{ and } k = 0,1,\dots, x-1 \}
\end{align*}
which is the length of the number of transitions in the shortest path from $q_n$ to an accepting state.
\end{definition}

Then we also have a product automaton with distance, $\mathcal{A}_{p,d} = \mathcal{T} \otimes \mathcal{A}_\varphi = (Q', \delta', Q_0', \mathcal{F}', W_p, d_p)$, defined similarly, with $d_p(q') = d(q'|\Q)$. We will refer to $q'$ as being on level $n$ if $d_p(q') = n$.

The idea of our algorithm is to start from $q_0' \in \Q'$, say $d_p(q_0')=n$ and then use a Dijkstra search to find the closest node that is on next smallest level, $n-1$. Then we will do another Dijkstra search on the next level down to find the closest node that has a transition down, and so on. This ensures that we will approach the accepting states i.e.\ those states on level 0. 

Once we reach an accepting state, we must use a Dijkstra search. This is because the idea of decreasing levels cannot be used to find a specific accepting state, simply an accepting state. Therefore we need to use a Dijkstra search which will search through all the nodes till it finds the accepting state we are looking for. We will refer to the run generated by this algorithm as $R_{new}$.
\begin{algorithm}
\caption{NearestNeighborRun()}\label{NNrun}
\begin{algorithmic}[1]
\Require Input $\A_{p,d}, S' = \Q_0'$ by default
\Ensure $R_{nn}$
%\Procedure{MyProcedure}{}
\State Level = $d_p(q_0' \in \Q_0')$, Prefix = [$q_0'$]
\While {Level > 0}
\State find the closest node, NextNode, that is on level one less than Level %using \texttt{adapted\_dijkstra\_multisource}
\State	add path to NextNode onto Path
\If {$d_p(\text{NextNode})$ == 0 }
\State break
\EndIf
\State Level = Level - 1	
\EndWhile
\State Suffix = use Dijkstra search to find the optimal path from NexNode back to itself %\texttt{DijksCycle($A_{p,d}$,NextNode)}
\State $R_{nn}$, prefix + suffix.
\end{algorithmic}
\end{algorithm}

Algorithm \ref{NNrun} is equivalent to the function \texttt{Garrett\_search} which is provided in the appendix. This code was based on \texttt{dijkstra\_plan\_networkX} from \cite{pMasGit} and still shares some of the structure. Finding the closest node on the level below the current level, i.e.\ NextNode is done using the function \texttt{adapted\_dijkstra\_multisource} which is also included in the appendix. This code was based on the function  \texttt{\_dijkstra\_multisource} in \cite{schult08}. When we provide computation runs of our algorithm in the following text, we will be referring to runs done with this algorithm. %Again, the script used to do these runs is also provided. 

As we can see, assuming that we reach an accepting state in a strongly connected component, we will do $n+1$ searches. This still may seem like a lot, however the searches are done on much smaller graphs. The first $n$ searches only look at graphs with $|\Pi|$ nodes. These smaller graphs have a number of edges less than or equal to $|\delta|$ i.e.\ the number of edges $\T$ has. This is because $\langle \pi_j, q_n \rangle \in \delta' (\langle \pi_i, q_m \rangle )$ iff $(\pi_i , \pi_j ) \in \rightarrow_c$ and $q_n \in \delta (q_m, L_c(pi))$, which implies the number of edges on one level is less than or equal to $|\delta|$. Therefore our worst case complexity will be $\mathcal{O}(|\delta|\cdot \log |\T| \cdot n) + \mathcal{O}(|\delta'| \cdot \log|\Q'|) = \mathcal{O}(|\delta|\cdot \log |\T| \cdot n + |\delta'| \cdot \log|\Q'|)$ where $n$ is the level of the initial node. This complexity applies in the situation that we find an accepting node, the accepting node we find has a path back to itself, and that we do not have transfers on the same level of the B\"uchi automaton i.e.\ if there is a transfer from $q_i$ to $q_{i+1}$ $\rightarrow$ $d(q_i) \neq d(q_{i+1})$. %However, when we get to examples in the complex formulas chapter this is not the case. If this distance requirement is not fulfilled, the worst case complexity of our algorithm is the same as the worst case complexity of the accepted algorithm i.e.\ $\mathcal{O}(|\delta'| \cdot \log|\Q'| )$.  


We now analyse how this algorithm performs in when the LTL formula expresses certain behaviours. 