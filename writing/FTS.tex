\section{Abstraction of the Workspace}
In \cite{belta07}, Belta et al describe robot path planning as consisting of three parts: the specification level, execution level, and the implementation level. The first level, the specification level, involves creating a graph (B\"{u}chi automata, which will be defined later) that represents the robot motion. Next is the execution level, which involves finding a path through the graph that satisfies a specification. Lastly, in the implementation level robot controllers are constructed that satisfy the path found in the previous step. 

We assume that we have one robot which is located in a given workspace denote $W_0 \subset \mathbb{R}^n$, which is bounded. To create a graph that represents the robot motion we need to consider the workspace along with the dynamics of the robot. To represent our workspace, which is a subspace of $\mathbb{R}^n$, in a finite graph we must partition it into a finite number of equivalence classes. A partition map is formally defined in definition \ref{def:PM}. Any partition can be used as long as it satisfy the bisimulation property, which will be defined later once more notation has been introduced \cite{belta04}. We denote the $\Pi = {\pi_1, \pi_2, \dots, \pi_w}$ to be the set of equivalence classes the workspace has been partitioned into, and thus $\cup_{i=1}^w \pi_i = W_0$ and $\pi_i \cap \pi_j = \emptyset$, $\forall i,j=1,2,\dots,w$ and $i\neq j$. We will henceforth refer to equivalence class $\pi_i$ as region $\pi_i$ for $i = 0,1,\dots, w$. 

\begin{definition}
\label{def:PM}
A partition map, $T: W_0 \rightarrow \Pi$ sends each state $x \in W_0$ to the finite set of equivalence classes $\Pi = {\pi_i}$,  $i = 1,2,\dots ,w$. $T^{-1}(\pi_i)$ is then all the states $x \in W_0$ that are in the equivalence class $\pi_i$ \cite{fainekos05}. 
\end{definition} 

We now introduce atomic propositions, which will be the building blocks for our task specification. Atomic propositions are boolean variables, and will be used to express properties about the state or the robot or the workspace. We define the following set of atomic propositions $AP_r = {a_{r,i}}$, $i=1,2,\dots,w$ where 
\[\alpha_{r,i} =  \begin{cases}
\top & \text{if the robot is in region $\pi_i$} \\
\bot & \text{else}
\end{cases}
\]
which represent the robot's location \cite{guo15}. Other things we want to be able to express are potential tasks, denote $AP_p = {\alpha_{p,i}}$, $i=1,2,\dots,m$. These can be statements such as "there is a ball in region $\pi_I$" or "the robot beeps"
We now define the set of all propositions $AP = AP_r \cup AP_p$.

\theoremstyle{definition}
\begin{definition}
\label{defCLF}
The labelling function $L_C:W_0 \rightarrow 2^{AP}$ maps a point $x \in W_0$ to the set of atomic propositions satisfied by $x$ \cite{guo15}.
\end{definition} 

We also include a definition of the discrete counterpart 

\theoremstyle{definition}
\begin{definition}
\label{defDLF}
The labelling function $L_D:\Pi\rightarrow 2^{AP}$ maps a region $\pi_i \in \Pi$ to the set of atomic propositions satisfied by $\pi_i$.
\end{definition} 
Note: $2^{AP}$ is the powerset of $AP$, i.e. the set of all subsets of $AP$ include the null set and $AP$.

For example, by definition, $a_{r,i} \in L_D(\pi_i)$. 

To define a graph that represents our environment, we must also consider the dynamics of the robot. The dynamics are relevant because they define the relationship between the various regions. The relationship we refer to is known as a transition. We define a transition between two points in $W_0$ as follows
\theoremstyle{definition}
\begin{definition}
\label{defCTransition}
There is a continuous transition, $\rightarrow_C \subset W_0 \times W_0$ from $x$ to $x'$, denoted $x \rightarrow_C x'$ if it is possible to construct a trajectory $x(t)$ for $0 \leq t \leq T$ with $x(0)=x$ and $x(T) =x'$ and we have $x(t) \in (T^{-1}(T(x))\cup T^{-1}(T(x')))$ \cite{fainekos09}
\end{definition}

We then say that there is a transition between two regions if from any point in the first region there is a transition to a point in the second region. More formally

\theoremstyle{definition}
\begin{definition}
\label{defDTransition}
There is a discrete transition, $\rightarrow_D \subset \Pi \times \Pi$, from $\pi_i$ to $\pi_j$, denoted $\pi_i \rightarrow_D \pi_j$ if there exists $x$ and $x'$ such that $T(x) = \pi_i$, $T(x')=\pi_j$ and $x \rightarrow_C x'$
\end{definition}

We can now define bisimulations
\theoremstyle{definition}
\begin{definition}
\label{defDTransition}
A partition $T:W_0\rightarrow \Pi$ is called a bisimulation if the following properties hold for all $x,y \in W_0$
\begin{enumerate}
    \item (Observation Preserving): If $T(x)=T(y)$, then $L_C(x) = L_C(y)$.
    \item (Reachability Preserving): If $T(x) = T(y)$, then if $x\rightarrow_C x'$ then $y \rightarrow_C y'$ for some $y'$ with $T(x')=T(y')$
\end{enumerate}
\end{definition}

The Observation Preserving requirement makes sure we do not allow the situation where part of $\pi_i$ fulfils $\alpha \in AP$ while part of $\pi_i$ does not, and the Reachability Preserving requirement ensures that for every point in region $\pi_i$, there exists a trajectory to some point $x'$, such that $T(x') = \pi_j$ if $\pi_i \rightarrow_D \pi_j$. These two requirements together guarantee that the discrete path we compute is feasible at the continuous level.


We can now define Finite-State Transition System (FTS), which is how we will represent our workspace.
\theoremstyle{definition}
\begin{definition}
\label{defFTS}
An FTS is a tuple $\mathcal{T}_C = (\Pi, \rightarrow_D, \Pi_0, AP,L_C)$ where $\Pi$ is the set of states, $\rightarrow_D \subseteq \Pi \times \Pi$ is the transitions relation where $(\pi_i,\pi_j) \in \rightarrow_C$ iff there is a transition from $\pi_i$ to $\pi_j$ as defined in definition \ref{defTransition}. In adherence to common notation, we will write $\pi_i \rightarrow_C \pi_j$. Note: $\pi_i \rightarrow_C \pi_i, \hspace{0.2cm} \forall 1,2,\dots w$. $\Pi_0 \subseteq \Pi$ is the initial state(s), $AP=AP_r \cup AP_p$ is the set of atomic propositions, and $L_C: \Pi \rightarrow 2^{AP}$ is the labelling function defined in definition \ref{defLF}.
\end{definition}

In this thesis, we will also consider the weighted FTS (WFTS)
\begin{definition}
\label{defFTS}
A WTFS is a tuple $\mathcal{T}_C = (\Pi, \rightarrow_C, \Pi_0, AP,L_C,W_C)$ where $\Pi$, $\rightarrow_C$, $\Pi_0$, $AP$, and $L_C$ are defined as in definition \ref{defFTS} and $W_C: \Pi \times \Pi \rightarrow \R^+$ is the weight function i.e.\ the cost of a transition in $\rightarrow_C$. 
\end{definition}
Note: Any FTS can be written can be converted to an WFTS with the weights of all transitions equalling one.

We use the FTS which represents our workspace to search for paths that are doable for our robot. When we search for a path, from one state we will only consider states which have a transition from our current state, because these are the only states the robot can move to. When talking about FTS, it can be helpful to use the notation $\Pre(\pi_i) = \{\pi_j \in \Pi | \pi_j \rightarrow_D \pi_i\}$ to define the the predecessors of the state $\pi_i$ and $\Post(\pi_i) = \{\pi_j \in \Pi | \pi_j \rightarrow_D \pi_i\}$ to define the the successors of the state $\pi_i$. In this thesis, we will deal with infinite paths. An infinite path is an infinite sequence of states $\tau = \pi_1 \pi_2 \dots$ such that $\pi_i \in \Pi_0$ and $\pi_i \in $ Post($\pi_{i-1}) \hspace{0.2cm} \forall i > 0$. The trace of a path is the sequence of sets of atomic propositions that are true in the states along a path i.e.\ trace($\tau$) = $L_D(\pi_1)L_D(\pi_2) \dots$.  


