\section{Linear Temporal Logic (LTL)}
To define tasks for our robot we must choose a high level language. Temporal logics are especially suited for defining robot tasks because of their ability to express not only fomulas constructed of atomic propositions and standard boolean connectives (conjunction, disjunction, and negation), but also temporal specifications e.g.\ $\alpha$ is true at some point of time. The particular temporal logic we will be using is known as linear temporal logic (LTL) \cite{clarke99}. LTL formulas are defined over a set of atomic propositions $AP$ according to the following grammar:

\begin{align*}
    \varphi ::= \top | \alpha | \neg \varphi_1 | \varphi_1  \lor \varphi_2 | \textbf{X} \varphi_1 | \varphi_1 \bm{\mathcal{U}} \varphi_2
\end{align*}

where $\top$ is the predicate true, $\alpha \in AP$ is an atomic proposition, $\varphi_1$ and $\varphi_2$ are LTL fomulas, $\neg$ and $\lor$ denote denote the standard Boolean connectives negation and disjunction respectively, X being the "Next" operator. $U$ is the temporal operator "Until", with $\varphi_1 \mathcal{U} \varphi_2$ meaning $\varphi_1$ is true until $\varphi_2$ becomes true. Given these operators, we can define the following additional prepositional operators:
\begin{align*}
    \text{Conjunction: }&  \varphi_1  \land \varphi_2 = \neg(\neg \varphi_1 \lor \neg \varphi_2) \\
    \text{Implication: }& \varphi_1 \Rightarrow \varphi_2 = \neg \varphi_1 \lor \varphi_2 \\
    \text{Equivalence: }& \varphi_1 \Leftrightarrow \varphi_2 = (\varphi_1 \Rightarrow \varphi_2) \land (\varphi_2 \Rightarrow \varphi_1)
\end{align*}
We note quickly that we have the false predicate, $\bot = \neg \top$.
We are also able to derive the following additional temporal operators:
\begin{align*}
    \text{Eventuality: }& \diamond \varphi_1 = T U \varphi_1 \\
    \text{Always: }& \square \varphi_1 = \neg \diamond \neg \varphi_1
\end{align*}

There is a growing interest in path and mission planning in robots using temporal logic specifications given the easy extension from natural language to temporal logic. We now give examples to illustrate this point. First, the atomic operators generally capture properties of the robot or the environment i.e. "the robot is in region 1", "the ball is in region 2", "the robot is holding a ball". We now give examples of common tasks converted to LTL formulas \citep{fainekos09} 
\begin{enumerate}
    \item \textbf{Reachability while avoiding regions}: "Go to region $\pi_{n+1}$ while avoiding regions $\pi_1, \pi_2, \dots, \pi_n$" \\ $\neg(\pi_1 \lor \pi_2 \dots \pi_n) \mathcal{U} \pi_{n+1}$ 
    \item \textbf{Sequencing}: "Visit regions $\pi_1, \pi_2, \pi_3$ in that order"\\ 
    $\diamond (\pi_1 \land \diamond(\pi_2 \land \diamond \pi_3))$ 
    \item \textbf{Coverage}: "Visit regions $\pi_1, \pi_2, \dots, \pi_n$ in any order"\\ $\diamond \pi_1 \land \diamond \pi_2 \land \dots \land \diamond \pi_n$
    \item \textbf{Recurrence (Liveness)}: "Visit regions $\pi_1, \dots, \pi_n$ in any order over and over again"\\ $\square(\diamond \pi_1 \land \diamond \pi_2 \land \dots \land \diamond \pi_n)$      
\end{enumerate}
Of course more complicated tasks are also expressible in LTL, and atomic propositions need not only refer to the location of the robot. Here is an example given in  \cite{guo15}: "Pick up the red ball, drop it to one of the baskets and then stay in room one" \\
$\diamond(rball \land \diamond basket) \land \diamond \square r1$ 

As it is possible to develop develop computational interfaces between natural language and temporal logics\cite{kress07}, LTL.

An infinite word over the alphabet $2^{AP}$ is an infinite sequence $\sigma \in (2^{AP})^\omega$, i.e.\ $\sigma = S_0 S_1 S_2 \dots$ where $S_k \in 2^{AP}$ for all $k=1,2,\dots$ where $S_k$ is the set of atomic propositions that are true at the time stop $k$. 

\theoremstyle{definition}
\begin{definition}
\label{defLTLS}
The semantics of LTL are defined as follows:
\begin{align*}
(\sigma,k) \models \alpha \hspace{0.3cm}\text{ if }\hspace{0.3cm}& \alpha \in S_k \\
(\sigma,k) \models \neg \varphi \hspace{0.3cm}\text{ if }\hspace{0.3cm}& (\sigma, k) \not \models \varphi \\
(\sigma,k) \models \textbf{X} \varphi \hspace{0.3cm}\text{ if }\hspace{0.3cm}& (\sigma, k+1) \models \varphi \\
(\sigma,k) \models \varphi_1 \lor \varphi_2 \hspace{0.3cm}\text{ if }\hspace{0.3cm}& (\sigma,k) \models \varphi_1 \text{ or } (\sigma,k) \models \varphi_2 \\
(\sigma,k) \models \varphi_1 \mathcal{U} \varphi_2 \hspace{0.3cm}\text{ if }\hspace{0.3cm}& \exists k' \in [k,+\inf ], \hspace{0.1cm} (\sigma ,k') \models \varphi_2 \text{ and } \\ &\forall k'' \in (k,k'), \hspace{0.1cm} (\sigma, k'') \models \varphi_1 
\end{align*}
\end{definition}
In the interest of the reader, we will denote $(\sigma ,0) \models \varphi$ by $\sigma \models \varphi$, which is what is used for all this thesis. The connection to an infinite path $\tau$ of an FTS is that the trace of the path, trace($\tau$), is a word over the alphabet $2^{AP}$. Given the LTL semantics, we can now verify if a path satisfies an LTL formula! We will say an infinite path $\tau$ satisfies $\varphi$ if its trace satisfies $\varphi$, i.e.\ $\tau \models \varphi$ if $trace(\tau) \models \varphi$. A path satisfying $\varphi$ will be referred to as a plan for $\varphi$.



We now describe how to construct a path of an FTS that satisfies an LTL formula. 

 

